% -----------------------------------------------------------------
% => ABSTRACT [DOING]
% -----------------------------------------------------------------
% 
\begin{resumo}

Atualmente, nós vivemos em uma sociedade tecnológica. Por essa razão, propor novas metodologias de ensino, que integrem recursos tecnológicos, que pressuponham o uso de inteligência artificial, internet das coisas e sistemas ciberfísicos, com a finalidade de aprimorar as experiências pedagógicas e, consequentemente, a qualidade da educação é uma demanda socialmente importante. Assim, no contexto da Educação 4.0, esta tese discorre sobre a resolução de um problema que pode ser expresso através da seguinte pergunta: \textit{é possível a construção de um ambiente de educação suportado por tecnologia, que utilize recursos computacionais físico-digitais como parte integrante do processo de ensino-aprendizagem e que, adicionalmente, proveja elementos que auxiliem na avaliação e acompanhamento dos estudantes?}
%minimize a dificuldade dos professores em inserir recursos computacionais físico-cibernéticos como parte integrante da metodologia de ensino e que, adicionalmente, proveja análises, inferências e recomendações para a melhoria do processo ensino-aprendizagem?}
A proposta desta tese consiste em apresentar uma plataforma de educação apoiada por tecnologia que integre manipulativos físico-digitais e métricas de avaliação da aprendizagem e, analisar como manipulativos físico-digitais criados através do modelo proposto podem auxiliar no processo e acompanhamento do ensino-aprendizagem.
%proveja recomendações de recursos educacionais baseadas em análise inteligente de aprendizagem visando atender às diversas demandas de aprendizagem dos estudantes. 
Desse modo, foi projetada uma plataforma composta de quatro partes: (i) Compositor: componente responsável pela geração de material didático, incluindo a inserção de objetos tangíveis de aprendizagem; (ii) Player Físico-Virtual: interface multiplataforma baseada em web responsável pela execução da parte digital da aula, incluindo possibilidades de interação com objeto tangível; (iii) Servidor: cria e gerencia a sala de aula virtual, trocando mensagens com todos os dispositivos do ambiente de aprendizagem, inclusive os diversos componentes físico-digitais dos recursos pedagógicos; e, (iv) Analíticos: componente responsável por calcular as diversas métricas de aprendizagem propostas a fim de fornecer informações e análises baseadas na interação do estudante com o material didático. 
A avaliação experimental foi feita da seguinte forma: (a) análises de um estudo exploratório das métricas de aprendizagem propostas cujo objetivo foi de verificar as análises que tais métricas poderiam prover; (b) Estudo de Caso dividido em três fases com o objetivo de: (i) analisar o impacto na aprendizagem do uso do objeto tangível construído de acordo com o modelo proposto; (ii) verificar a percepção dos estudantes com relação ao uso do objeto tangível; e, por fim, (iii) explorar a viabilidade do uso de um objeto tangível para avaliação/acompanhamento da aprendizagem no contexto da plataforma proposta. 

\novo{Após a escrita do capítulo de Resultados, complementar com o que foi achado `findings'}


%Além da proposta, como resultados, são apresentados: (a) análises de experimentos prévios cujo objetivo foi de verificar as análises que as métricas propostas poderiam prover; (b) dois mapeamentos sistemáticos da literatura, onde o primeiro abordou a recomendação de recursos educacionais baseados em computação cognitiva e o segundo abordou a criação de ambientes de aprendizagem baseados em ambientes ciber-físicos; e (c) uma proposta inicial de um estudo de caso que sirva de prova de conceito para o método apresentado.

\end{resumo}