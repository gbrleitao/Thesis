\chapter{Perfil de aplicação OBAA} \label{Chap:AppendixA}


\begin{xltabular}{\textwidth}{|l|l|X|}
	\hline \multicolumn{1}{|c|}{\textbf{Item}} & \multicolumn{1}{c|}{\textbf{Nome}} & \multicolumn{1}{c|}{\textbf{Comentário}} \\ \hline
	\endfirsthead
	
	\hline \multicolumn{3}{|c|}{continuação da página anterior} \\ 
	\hline \multicolumn{1}{|c|}{\textbf{Item}} & \multicolumn{1}{c|}{\textbf{Nome}} & \multicolumn{1}{c|}{\textbf{Comentário}} \\ \hline
	\endhead
	
	\hline \multicolumn{3}{|r|}{Continua na próxima página} \\ \hline

	\endfoot
	
	\endlastfoot	
%N & \textbf{Nome} & \textbf{Comentário} \\ \hline
\textbf{1} & General & Esta categoria agrupa as informações gerais que   descrevem esse objeto de aprendizado como um todo. \\ \hline
\textbf{1.1} & General.Identifier & Rótulo globalmente exclusivo que identifica esse Objeto de Aprendizagem \\ \hline
\textbf{1.1.1} & General.Identifier.Catalog & O nome ou designador do esquema de identificação ou catalogação para esta entrada. Um esquema de namespace. \\ \hline
\textbf{1.1.2} & General.Identifier.Entry & O valor do identificador dentro do esquema de identificação ou catalogação que designa ou identifica esse objeto de aprendizado. Uma cadeia específica de namespace. \\ \hline
\textbf{1.2} & General.Title & Nome dado ao OA \\ \hline
\textbf{1.3} & General.Language & A principal linguagem humana ou linguagens usadas nesse objeto de aprendizado para se comunicar com o usuário pretendido. \\ \hline
\textbf{1.4} & General.Description & Descrição textual do conteúdo do OA \\ \hline
\textbf{1.5} & General.Keyword & Palavras-chave associadas ao OA \\ \hline
\textbf{1.6} & General.Coverage & Abrangência do OA: localização espacial, período temporal, jurisdição. (http://www.dublincore.org/documents/dces/) \\ \hline
\textbf{1.7} & General.Structure & Estrutura organizacional do OA \\ \hline
\textbf{1.8} & General.AggregationLevel & Granularidade funcional do OA \\ \hline
\textbf{2} & LifeCycle & Grupo de metadados que contém informações sobre o histórico e o estado atual do OA \\ \hline
\textbf{2.1} & LifeCycle.Version & A Edição do OA \\ \hline
\textbf{2.3} & LifeCycle.Contribute & Entidades (ou seja, pessoas, organizações) que contribuíram para o estado desse OA durante seu ciclo de vida (por exemplo, criação, edições, publicação). \\ \hline
\textbf{2.3.1} & LifeCycle.Contribute.Role & Tipo de contribuição. 

NOTA - Minimamente, o(s) autor (es) do OA deve ser descrito.\\ \hline
\textbf{2.3.2} & LifeCycle.Contribute.Entity & Identificação e informação sobre entidades (ou seja, pessoas, organizações) contribuindo para este objeto de aprendizagem. 

As entidades devem ser classificadas como mais relevantes primeiro. \\ \hline
\textbf{2.3.3} & LifeCycle.Contribute.Date & Data de contribuição \\ \hline
\textbf{3} & Meta-Metadata & Grupo de metadados que contém informações sobre os metadados do objeto \\ \hline
\textbf{3.1} & Meta-Metadata.Identifier & Identificador deste registro de metadados \\ \hline
\textbf{3.2} & Meta-Metadata.Contribute & Entradas relacionadas à criação e modificações dos metadados do objeto \\ \hline
\textbf{3.4} & Meta-Metadata.Language & Linguagem/idioma utilizada no metadado em si. \\ \hline
\textbf{4} & Technical & Informações relacionadas aos requisitos e características técnicas dos objetos de aprendizagem \\ \hline
\textbf{4.1} & Technical.Format & Formato dos conteúdos do objeto \\ \hline
\textbf{4.2} & Technical.Size & Tamanho do objeto de aprendizagem em bytes \\ \hline
\textbf{4.3} & Technical.Location & Localização física do objeto e de seus conteúdos (URL, URI...) \\ \hline
\textbf{4.4} & Technical.Requirement & Requisitos técnicos para uso do objeto de aprendizagem. (Se houver vários requisitos, todos são necessários, ou seja, o conector lógico é AND) \\ \hline
\textbf{4.6} & Technical.OtherPlatformsRequirements & Informações sobre outras plataformas e hardwares \\ \hline
\textbf{4.7} & Technical.Duration & Duração pretendida para a exibição do objeto. 

OBS:  este elemento é especialmente utilizado para áudios, vídeos ou animações. \\ \hline
\textbf{4.8} & Technical.SupportedPlatforms & Lista de plataformas digitais para as quais o Objeto de Aprendizagem está previsto. Atualmente estão previstos três tipos básicos de plataformas digitais para disponibilização de OAs: Web, DTV e Mobile. 

Este item não é obrigatório, para manter a compatibilidade com o LOM, mas é recomendado seu preenchimento. \\ \hline
\textbf{4.9} & Technical.PlatformSpecificFeatures & Conjunto de características técnicas das mídias específicas desenvolvidas para cada plataforma para a qual o OA foi previsto. \\ \hline
\textbf{5} & Educational & Grupo de metadados que descreve as características educacionais e pedagógicas do OA. 

NOTA - Esta é a informação pedagógica essencial para aqueles envolvidos em alcançar uma experiência de aprendizagem de qualidade. O público para esses metadados inclui professores, gerentes, autores e alunos. \\ \hline
\textbf{5.1} & Educational.InteractivityType & Modo predominante de aprendizagem apoiado por este projeto de aprendizagem. 

O aprendizado ``ativo'' (por exemplo, aprender fazendo) é apoiado por conteúdo que induz diretamente a ação produtiva do aluno. Um objeto de aprendizado ativo solicita ao aprendiz informações semanticamente significativas ou para algum outro tipo de ação ou decisão produtiva, não necessariamente executada na estrutura do objeto de aprendizagem. Documentos ativos incluem simulações, questionários e exercícios. 

A aprendizagem ``expositiva'' (por exemplo, aprendizagem passiva) ocorre quando o trabalho do aluno consiste principalmente em absorver o conteúdo exposto a ele (geralmente por meio de texto, imagens ou som). Um objeto de aprendizagem expositivo exibe informações, mas não solicita ao aluno qualquer entrada semanticamente significativa. Documentos expositivos incluem ensaios, clipes de vídeo, todos os tipos de material gráfico e documentos em hipertexto. Quando um objeto de aprendizado combina os tipos de interatividade ativa e expositiva, seu tipo de interatividade é ``misto''. 

OBSERVAÇÃO: Ativar links para navegar em documentos de hipertexto não é considerado uma ação produtiva. \\ \hline
\textbf{5.2} & Educational.LearningResourceType & Tipo específico do objeto de aprendizagem.

Tipos: Exercício, Simulação, Questionário, Diagrama, Figura, Gráfico, Índice, Apresentação, Tabela, Texto Narrativo, Exame, Experimento, Solução de Problemas, Auto-avaliação, Palestra.

 \\ \hline
\textbf{5.3} & Educational.InteractivityLevel & O grau de interatividade que caracteriza este objeto de aprendizagem. Interatividade neste contexto refere-se ao grau em que o aluno pode influenciar o aspecto ou comportamento do objeto de aprendizagem. 

NOTA - Inerentemente, esta escala é significativa dentro do contexto de uma comunidade de prática.

Níveis: Muito baixo, Baixo, Médio, Alto, Muito Alto.

 \\ \hline
\textbf{5.5} & Educational.IntendedEndUserRole & Principal usuário (s) para o qual este objeto de aprendizagem foi projetado, mais dominante primeiro. 

NOTAS 

1 - Um aluno trabalha com um objeto de aprendizado para aprender alguma coisa. 

Um autor cria ou publica um objeto de aprendizado. 

Um gerente gerencia a entrega desse objeto de aprendizado, por exemplo, uma universidade ou faculdade. O documento para um gerente é tipicamente um currículo. 

2 - Para descrever o papel do usuário final pretendido através das habilidades que o usuário pretende dominar, ou as tarefas que ele ou ela se destina a realizar, a categoria 9: Classificação pode ser usada. \\ \hline
\textbf{5.6} & Educational.Context & O principal ambiente dentro do qual a aprendizagem e o uso desse objeto de aprendizado se destinam a ocorrer. 

NOTA - A boa prática sugerida é usar um dos valores do espaço de valor e usar uma instância adicional desse elemento de dados para refinamento adicional, como em (``LOMv1.0'', ``ensino superior”) e (``http://www.ond.vlaanderen.be/onderwijsinv

laanderen/Default.htm”,“kandidatuursonderw

ijs'') 


Contextos: Escola, Ensino Superior, Estágio, Outros

\\ \hline
\textbf{5.7} & Educational.TypicalAgeRange & Idade típica dos principais usuários do OA. 

Este elemento de dados deve referir-se à idade de desenvolvimento, se isso for diferente da idade cronológica. 

NOTAS
1 - A idade do aluno é importante para encontrar OA, especialmente para os alunos em idade escolar e seus professores. 

Quando aplicável, a string deve ser formatada como idade mínima - idade máxima ou idade mínima -. 

{[}Este é um compromisso entre adicionar três elementos componentes (idade mínima, idade máxima e descrição) e ter apenas um campo de texto livre.{]} 

2 - Esquemas alternativos para o que este elemento de dados tenta cobrir (como vários esquemas de leitura de idade ou nível de leitura, QI ou medidas de idade de desenvolvimento) devem ser representados através da categoria 9: Classificação. \\ \hline

\textbf{5.8} & Educational.Difficulty & Grau de dificuldade de se trabalhar com o OA para o público-alvo típico pretendido. \\ \hline
\textbf{5.9} & Educational.LearningTime & Tempo aproximado ou típico necessário para trabalhar com ou através desse objeto de aprendizagem para o público-alvo pretendido. \\ \hline
\textbf{5.10} & Educational.Description & Comentários sobre como esse objeto de aprendizagem deve ser usado. \\ \hline
\textbf{5.11} & Educational.Language & Linguagem natural usada pelo usuário típico do Objeto de Aprendizagem. \\ \hline
\textbf{5.12} & Educational.LearningContentType & Especificação educacional do tipo de conteúdo do objeto de aprendizagem. \\ \hline
\textbf{5.13} & Educational.Interaction & Especifica a interação educacional proposta por este objeto de aprendizagem e seu(s) usuários. \\ \hline
\textbf{5.14} & Educational.DidacticStrategy & Conjunto de ações planejadas e conduzidas pelo professor a fim de promover o envolvimento e comprometimento dos alunos com um conjunto maior de atividades. \\ \hline
\textbf{6} & Rights & Esta categoria descreve os direitos de propriedade intelectual e as condições de uso desse objeto de aprendizado. 

OBSERVAÇÃO: a intenção é reutilizar os resultados do trabalho em andamento nas comunidades de direitos de propriedade intelectual e comércio eletrônico.

Atualmente, esta categoria fornece somente o nível mínimo de detalhes.\\ \hline
\textbf{6.1} & Rights.Cost & Se o uso deste OA requer pagamentos \\ \hline
\textbf{6.2} & Rights.CopyRightandOtherRestrictions & Se direitos autorais ou outras restrições se aplicam ao uso desse objeto de aprendizado. \\ \hline
\textbf{6.3} & Rights.Description & Comentários sobre as condições de uso deste Objeto de Aprendizagem \\ \hline
\textbf{7} & Relation & Esta categoria define a relação entre esse objeto de aprendizado e outros objetos de aprendizado, se houver. 

Para definir vários relacionamentos, pode haver várias instâncias dessa categoria. 

Se houver mais de um objeto de aprendizado de destino, cada destino terá uma nova instância de relacionamento. \\ \hline
\textbf{7.2} & Relation.Resource & OA alvo que este OA referencia \\ \hline
\textbf{7.2.1} & Relation.Resource.Identifier & Marcador global único que identifica o Objeto de Aprendizagem alvo \\ \hline
\textbf{8} & Annotation & Essa categoria fornece comentários sobre o uso educacional desse objeto de aprendizado e informações sobre quando e por quem os comentários foram criados. 

Esta categoria permite que os educadores compartilhem suas avaliações de objetos de aprendizagem, sugestões de uso, etc.\\ \hline
\textbf{8.1} & Annotation.Entity & Entidade que criou esta anotação (pessoa, organização...) \\ \hline
\textbf{8.2} & Annotation.Date & Data que esta anotação foi criada \\ \hline
\textbf{8.3} & Annotation.Description & Conteúdo da Anotação \\ \hline
\textbf{10} & Acessibility & Acessibilidade é a habilidade do ambiente de aprendizagem de se adaptar às necessidades de cada usuário/estudante. Ela é determinada pela flexibilidade de um ambiente educacional (no que diz respeito à apresentação, métodos de controle, modalidade de acesso e suporte para os estudantes e a disponibilidade de conteúdos e atividades alternativas, mas equivalentes.\\ \hline
\textbf{11} & SegmentInformationTable & Grupo que conterá o conjunto de informações de segmentação dos objetos de aprendizagem e de grupos de segmentos dos objetos de aprendizagem.\\ \hline
\end{xltabular}