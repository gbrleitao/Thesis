
\chapter{Conclusão} \label{Chap:endFuture}

%Esta proposta de Tese apresentou e discutiu um modelo de educação apoiada por tecnologia que possibilita a utilização de objetos de aprendizagem que sejam simultaneamente físicos e virtuais e que possibilitem formas diferenciadas de avaliação da aprendizagem. 

Esta Tese apresentou e discutiu um modelo de educação apoiada por tecnologia que possibilita a criação e a utilização de objetos tangíveis de aprendizagem, além de formas diferenciadas de avaliação da aprendizagem. 
Dessa forma, foi apresentado um caminho para integração entre manipulativos tangíveis, ambientes digitais de aprendizagem e métricas de avaliação da aprendizagem de forma a possibilitar que uma plataforma digital de aprendizagem contenha, simultaneamente, objetos de aprendizagem tradicionais e tangíveis. Assim, neste capítulo, 
% são propostos diversos trabalhos que podem adicionar melhorias ao método aqui apresentado e 
são feitos comentários finais acerca do método e dos resultados obtidos.

\section{Considerações Finais}

%As novas tecnologias estão presentes em todos os níveis da sociedade atual, possibilitando maior capacidade de comunicação e facilitando diversos processos. Entretanto, pouco a pouco, os artefatos tecnológicos vão abandonando seu \textit{status} de simples ferramentas e passando a integrar a vida humana de tal maneira que todas as relações acabam sendo mediadas por alguma tecnologia. De certo modo, essa mediação tecnológica molda a maneira como o ser humano enxerga, interage e significa a realidade e o mundo ao seu redor.

%Se a maneira humana de significar o mundo e a realidade são profundamente marcadas pela tecnologia, então, o conhecimento e as formas de adquiri-lo ou construí-lo também o são. Sendo a escola o principal meio através do qual o indivíduo tem acesso ao conhecimento, os processos pedagógicos da educação escolar precisam ser revistos e atualizados.

Com as novas tecnologias cada vez mais presentes nos diversos processos sociais, culturais e econômicos atuais, propiciando especialmente maior agilidade e capacidade de comunicação, os artefatos tecnológicos vão marcando significativamente o mundo e as diversas relações estabelecidas entre o ser humano e a realidade, de modo que o conhecimento e as formas de adquiri-lo ou construí-lo também são afetados. Sendo a escola um dos principais meios através do qual os indivíduos tem acesso ao conhecimento, os processos pedagógicos da educação escolar precisam ser revistos e atualizados de modo que possam cumprir o seu papel social.

%É nessa perspectiva que este trabalho está inserido e, por causa disso, a pergunta levantada na seção de definição do problema (Seção~\ref{section:defProblem}) ressurge: ``é possível a construção de um ambiente de educação suportado por tecnologia, que minimize a dificuldade dos professores em inserir recursos computacionais físico-cibernéticos como parte integrante da metodologia de ensino e que, adicionalmente, proveja análises, inferências e recomendações para a melhoria do processo ensino-aprendizagem?''

É nessa perspectiva que este trabalho está inserido e, por causa disso, a pergunta levantada na seção de definição do problema (Seção~\ref{section:defProblem}) ressurge: ``é possível a construção de um ambiente de educação suportado por tecnologia que utilize recursos computacionais tangíveis como parte integrante do processo de ensino-aprendizagem e que, adicionalmente, proveja elementos que auxiliem na avaliação e acompanhamento dos estudantes?''

Com o objetivo de tentar responder a essa questão de fundo, foi aprimorado o método proposto na dissertação de mestrado de~\cite{Leitao:2014} de forma 
% e experimentado um método 
que o processo pedagógico seja apoiado em todas as suas fases através da inserção de tecnologia, de modo a apoiar os momentos de preparação, execução e após uma aula. Essa inserção de tecnologia 
deve permitir a utilização de uma metodologia pedagógica que seja suportada por artefatos tangíveis atuando como objetos de aprendizagem e que possibilitem uma avaliação da aprendizagem mais adequada.
% permitiu a utilização de uma metodologia de avaliação baseada em ajuda sob demanda, em que o estudante, em processo de construção de conhecimento, é o principal beneficiário.

Neste trabalho, embora haja uma proposta de métricas que incidam em uma aparente quantificação do aprendizado, é importante salientar que esta quantificação pode ser utilizada para investigar a qualidade do processo de aprendizado, inclusive, com vistas a ajudar na tomada de decisões e no desenvolvimento dos estudantes. Desse modo, a aplicação de métricas alternativas de acompanhamento da aprendizagem associadas ao uso de tecnologia pode colaborar na melhoria dos processos de aprendizagem. Um exemplo disso está na análise presente no estudo exploratório feito a partir da composição das métricas do Grau de Assertividade e Nível de Compreensão (Seção~\ref{sec:resultados_analytics}), em que foi possível identificar não apenas quais estudantes estavam tendo mais dificuldade, mas, também quais estavam mais inseguros. Além disso, tanto para esses estudantes, quanto para a turma inteira, a métrica Nota Ponderada associada a Nota Tradicional foram utilizadas para indicar quais tópicos e disciplinas precisam ser priorizados pelos estudantes na hora do estudo pessoal.

% do parâmetro Desvio feita na Seção~\ref{section:analise_compreensao}, onde observou-se mais detalhadamente o desempenho do estudante e foi verificado que, apesar de um desempenho ruim na métrica tradicional de Pontuação, 50\% das opções marcadas nas 10 questões principais (e consideradas incorretas pela métrica pontuação) correspondiam às alternativas mais próximas da correta, isto significa que este parâmetro, possibilita medir o nível de aprendizado de modo mais justo e possibilita ao professor oferecer melhor suporte para suprir as possíveis deficiências do aluno.

De modo geral, os resultados dos experimentos feitos até o momento permite intuir que a inserção de recursos tangíveis na aula possibilita a aplicação de diversas métricas de acompanhamento/avaliação da aprendizagem dos estudantes e abre caminho para o desenvolvimento de sistemas computacionais %de análise e recomendação automáticas 
que ofereçam %um suporte na escolha dos métodos e 
ferramentas pedagógicas que melhor dialoguem os estudantes e, assim, melhorem seu engajamento no processo de construção do conhecimento.

É necessário ressaltar que, ao longo do experimento do `Estudo de Caso' (Seção~\ref{sec:estudo_caso1}), observou-se um interesse e um engajamento maior dos participantes quando apresentados ao objeto tangível, em contraposição às atividades tradicionais executadas previamente. Tendo sido possível observar um maior engajamento no processo de aprendizagem, uma vez que na etapa tradicional pode-se constatar que alguns participantes estavam dispersos (ora jogando, ora navegando em sítios alheios ao conteúdo apresentado) e o mesmo não aconteceu durante o uso efetivo do objeto tangível. O engajamento e o interesse pelo objeto aconteceu de tal forma que, após o fim do experimento, o grupo de controle (Grupo A) pediu para fazer uso do mesmo, de modo que isso possibilitou as análises apresentadas na Seção~\ref{subsec:fase1-posteste2}.

Ademais, embora isso não tenha podido ser adequadamente mensurado, foi observado que alguns participantes possuíam um alto grau de defasagem com relação aos conteúdos prévios de matemática para o estudo do quadro trigonométrico (por exemplo: operações envolvendo `regra de três'), de modo que estes participantes também tiveram mais dificuldades tanto os que participaram do grupo de controle com ensino tradicional, quanto os que participaram do uso do objeto tangível. De toda forma, pode-se notar que o uso do objeto tangível proporcionou oportunidades de exercício e correlação dos conteúdos e, por consequência, de aprendizagem que o ensino tradicional não enfatizou.

Por fim, salienta-se que,
% este trabalho, propõe uma abordagem que permite certo grau de liberdade metodológica durante a fase de execução da aula, não obrigando o professor a optar por um ou outro paradigma de educação. Entretanto, 
a inserção de tecnologia tal como aqui é proposta pretende auxiliar no caminho rumo a uma educação que utilize os recursos computacionais como meio para construção do conhecimento,% o que implica em contribuir com um processo gradativo em direção a um paradigma construcionista, 
onde os dispositivos tecnológicos não são simples ferramentas de reprodução e transmissão de conteúdos, mas, meios de interpelar o estudante e instigá-lo na busca pelo conhecimento.

\section{Trabalhos Futuros}

A inserção de recursos computacionais nos processos de educação é apenas o ponto de partida para uma revolução nas metodologias de ensino e, principalmente, do acompanhamento da aprendizagem dos estudantes. Assim, uma série de trabalhos ainda precisam ser feitos, por exemplo, aumentar a amostra dos experimentos realizados neste trabalho de modo a se ter uma melhor confirmação da validação e depuração do método proposto nesta tese, em especial a criação e utilização de objetos tangíveis de aprendizagem associados às métricas de avaliação propostas.
% Assim, o próprio processo de validação do método proposto nesta dissertação, apontou trabalhos futuros para melhoria dos processos pedagógicos. Deste modo, como continuidade do desenvolvimento deste trabalho, propomos os seguintes pontos:

%Além disso, quando se fala em padronização dos objetos físico-virtuais de aprendizagem, uma série de desafios se fazem presentes, dentre esses desafios podemos elencar: (i) há uma demanda de como registrar as interações entre os estudantes e o ambiente físico-virtual; e, (ii) como efetivamente usar tais informações para avaliar a experiência de aprendizado e das condições do ambiente de ensino; (iii) há um problema de se fazer recomendações e avaliações desses objetos físico-virtuais uma vez que não há um repositório dos mesmos; (iv) há um problema para a criação e descrição de objetos de aprendizagem físico-virtuais porque não há um padrão de metadados e/ou ontologias; e (v) há um problema de sistematização da execução desses objetos, visto que os objetos criados não estão integrados a nenhuma plataforma ou a quaisquer ambientes de aprendizagem.

Além disso, quando se fala em padronização dos objetos tangíveis de aprendizagem, uma série de desafios se fazem presentes, dentre esses desafios podemos elencar: (i) há uma demanda de como registrar as interações entre os estudantes e o ambiente tangível, de modo que cada novo objeto solicita uma instrumentação com sensores diferenciada; e, (ii) como efetivamente usar tais informações para avaliar a experiência de aprendizado e das condições do ambiente de ensino, onde intuímos que o principal uso deve acontecer no sentido de fornecimento de novas \textit{features} para algoritmos de aprendizagem de máquina ou profunda; (iii) há um problema de se fazer recomendações e avaliações de objetos tangíveis uma vez que não há um repositório dos mesmos.

Obviamente, alguns destes problemas estão sendo abordados pelo método apresentado nesta tese, mas, carecem de maior aprofundamento e mais experimentos, que não foram possíveis devido às restrições de tempo, espaço e recursos impostos pela própria dinâmica de execução de um projeto de doutorado. 

Por fim, com relação às métricas de avaliação da aprendizagem, o caminho natural é o da automatização das análises no sentido de gerar diagnósticos e recomendações automáticas. E, no caso das recomendações, que sejam sugeridas mudanças metodológicas ou utilização de objetos de aprendizagem mais interessantes, ou ainda recomendações baseadas em ciber-física, por exemplo, supondo que o ambiente de aprendizagem que, munido das informações dos componentes curriculares estudados pelo aluno (virtual), tenha acesso a sua localização via GPS (físico), pode sugerir eventos extracurriculares que contribuam com o processo de aprendizado.