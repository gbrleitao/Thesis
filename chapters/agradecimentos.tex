\acknowledgements{
%Primeiramente, agradeço a vida e a saúde que me foram concedidas ao longo destes árduos tempos, foram muitas dificuldades, mas, todas superadas com paciência e auxílio de quem me acompanhou nessa jornada.

% Agradeço à minha família, meu esteio, sem os quais seria impossível eu ter chegado onde estou. Eu sou extremamente grato por todo o carinho, atenção e disponibilidade que vocês tem tido comigo ao longo dos anos.

%Especialmente, agradeço à minha esposa, Thais Augusto do Nascimento, pelo constante apoio e carinho, mas, principalmente, pela compreensão com as dificuldades inerentes a este processo e pelas discussões e ideias que muito contribuíram para o desenvolvimento desta pesquisa. Obrigado por estar comigo e me ajudar a abrir os olhos para as possibilidades dos diversos processos educativos.

%% Primeiramente, agradeço a Deus, sem o qual nada teria sido possível. Agradeço especialmente por tanto bem e tantas dádivas recebidas através das pessoas que me acompanharam, apoiaram e auxiliaram nesta pesquisa. Todas são sinais da presença amorosa de Deus na minha vida.

%%Primeiramente, agradeço a Deus por tanto bem e tantas dádivas recebidas através das pessoas que me apoiaram nesta pesquisa. Todas são sinais da Sua presença amorosa em minha vida. 

% Ao meu orientador e amigo, professor Raimundo da Silva Barreto, que tem acompanhado grande parte da minha odisseia acadêmica, desde os primeiros anos na graduação em Engenharia da Computação, minha temporada na Filosofia e, então, o meu retorno para continuar os estudos no Mestrado em Informática. Muito obrigado, professor, pela amizade, humanidade, orientação e, especialmente, por toda a confiança que tem depositado em mim e no meu processo.

% Às professoras Elaine Harada Teixeira de Oliveira e Francisca Cavalcante e aos professores Tiago Thompsen Primo e Juan Gabriel Colonna, muito obrigado pela solicitude em aceitar participar da minha banca de defesa e pelas excelentes contribuições seja por ocasião da qualificação, seja por outros projetos relacionados com os quais estivemos envolvidos.

% Agradeço também aos meus amigos e colegas do Grupo de Interesse em Sistemas Embarcados, do PPGI/ICOMP e dos cursos de Ciência da Computação, Engenharia de Software e Engenharia da Computação da UFAM que ajudaram imensamente na discussão e implementação deste trabalho. Faço um agradecimento especial a Anilton Carlos, Adelson Portela, Jeliel Augusto, Edson Magno, Edwin Juan, Fernando Furtado, Lucas Barreto, Timoteo Santos e Victor Augusto.

% Muito obrigado aos meus colegas de trabalho do IFAM que me apoiaram no afastamento para o doutorado, na viabilização dos experimentos ou segurando as pontas na preparação para a defesa da tese. Meu agradecimento especial a Ana Maria, Ana Paula, Eline, Fábio, Hilton, Jaidson, Lerkiane, Paulo Vitor, Valéria e Walter.

% Agradeço aos estudantes dos cursos técnicos integrado em informática e em jogos digitais do IFAM de Manacapuru que aceitaram participar e colaborar com a execução dos experimentos.

%Enfim, agradeço a todas as minhas amigas e amigos que tem me apoiado e torcido pelo meu sucesso pessoal e profissional.

%% Minha gratidão a CAPES pelo suporte financeiro para execução desta pesquisa. Parte dos resultados apresentados neste trabalho foram obtidos através do projeto de pesquisa ``Sistemas para Avaliação de Comportamento e Recomendação Inteligente em Ambientes Educacionais e de Saúde Remota'', financiado pela Samsung Eletrônica da Amazônia Ltda., no âmbito  da Lei no. 8.387 (art. 2º)/91.

}