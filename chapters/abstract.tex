% -----------------------------------------------------------------
% => ABSTRACT [DOING]
% -----------------------------------------------------------------
% 
\begin{abstract}
Coming soon...
%We live in a technological society, therefore we must propose new teaching methodologies that integrate modern technological resources and that suppose the use of artificial intelligence, internet of things and cyber-physical systems to improve pedagogical experiences and, consequently, the quality of education. Thus, in the context of Education 4.0, this thesis discusses the resolution of a problem that can be expressed through the following question: `\textit{is it possible to build a technology-supported education environment that minimizes the difficulty for teachers to insert physical-cyber computing resources such as essential part of learning methodology and that, in addition, provides analysis, inferences and recommendations to improvement of teaching-learning process?}' For this reason, the purpose of this thesis is to present a technology-supported education platform that enables the integration of learning objects that have at the same time, integrated physical and virtual components and which, after a classroom, provides recommendations of educational resources based on learning analytics to answer the diverse learning demands of students. Thus, a platform composed of five parts was designed: (i) Composer: component responsible for the generation of didactic material, including the insertion of physical-virtual learning objects; (ii) Physical-Virtual Player: runs the class with relative independence to the execution environment; (iii) Server: creates and manages the virtual classroom, exchanging messages with all devices in the learning environment, including the various physical and virtual components of pedagogical resources; (iv) Analytics: component responsible for calculating the learning metrics proposed in order to provide information and analysis based on student interaction with the didactic material; and (v) Intelligence: responsible for recommending educational resources and making notifications regarding student demand for study. Besides the proposal, as results, are presented: (a) analyzes of previous experiments whose objective was to verify the analyzes that the proposed metrics could provide; (b) two literature systematic mappings, where the first approached the recommendation of educational resources in cognitive computing-based and the second approached the creation of cyber-physics based learning environments; and (c) an initial proposal for a case study that will serve as proof of concept for the method presented.

\end{abstract}