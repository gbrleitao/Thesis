% -----------------------------------------------------------------
% => ABSTRACT [DOING]
% -----------------------------------------------------------------
% 
\begin{abstract}
We live in a technological society, therefore we must propose new teaching methodologies that integrate modern technological resources and that suppose the use of artificial intelligence, internet of things and cyber-physical systems to improve pedagogical experiences and, consequently, the quality of education. Thus, in the context of Education 4.0, this thesis discusses the resolution of a problem that can be expressed through the following question: `\textit{is it possible to build a technology-supported education environment that uses tangible computational resources as an integral part of teaching-learning process and that, in addition, provides elements that help in the evaluation and monitoring of students?}' The purpose of this work is to present one path to build a technology-supported education platform that enables the integration of tangible objects and metrics to learning evaluation and, to analyze how tangible manipulatives created using the proposed model can help in the teaching-learning process and monitoring. 
Thus, a platform composed of four parts was designed: (i) Composer: component responsible for the generation of didactic material, including the insertion of tangible learning objects; (ii) Tangible Player: runs the class with relative independence to the execution environment; (iii) Server: creates and manages the virtual classroom, exchanging messages with all devices in the learning environment, including the various physical and digital components of pedagogical resources; and (iv) Analytics: component responsible for calculating the learning metrics proposed in order to provide information and analysis based on student interaction with the didactic material;
The experimental evaluation was carried out as follows: (a) analysis of an exploratory study of proposed learning metrics whose objective was to verify the analyzes that could provide; (b) Case Study divided into three phases with the objective of: (i) analyzing the impact on learning the use of the tangible object built according to the proposed model; (ii) verify the students' perception regarding the use of the tangible object; and, finally, (iii) explore the feasibility of using a tangible object to assess/monitor the learning in context of the proposed platform. As a result, the approach using the tangible object built based on this proposal proved to be promising either in comparison with traditional teaching or in the integration with learning metrics. In addition, the tangible object used obtained acceptance percentages above $96\%$ in relation to perceived usefulness, above $88\%$ of perceived satisfaction and above $90\%$ of intention to use, according to the TAM3 form applied in Phase 2.

\textbf{Keywords:} 1. Tangible objects. 2. Education. 3. Learning environments. 4. Embedded Systems. 5. Learning manipulatives.
\end{abstract}